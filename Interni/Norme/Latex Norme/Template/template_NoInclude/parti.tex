%	Questo file è di esempio per le varie parti. Sono presenti alcuni metodi per scrivere il testo nel modo corretto con latex, dagli elenchi a tutto il resto.
%	Chiunque utilizzi comandi nuovi è pregato di riportarlo qui in modo che tutti ne conoscano l'effetto

%	\section è fondamentale da utilizzare per la creazione dell'indice. Section è la macro-parte, subsection tutte le altre. La numerazione di tutte le varie parti la fa Latex direttamente (parte 1, sottoparte 1.1,1.2 etc)
\section{Titolo del capitolo}{
	\subsection{sotto capitolo 1}{
		bla bla bla	 }
	\subsection{sotto capitolo 2}{ 
		bla bla bla	}
}

%	Formattazione testo e altro
\textbf{} % grassetto
\uppercase{} % maiuscolo
\emph{} % corsivo

%grandezza del carattere dal più piccolo al più grande
\tiny{} 
\small{}
\normalsize{}
\large{}
\Large{}
\huge{}
\Huge{}

%elenchi
%puntati
\begin{itemize}
	\item primo
	\item secondo
\end{itemize}
%numerati
\begin{enumerate}
	\item primo
	\item secondo
\end{enumerate}
%con descrizione: Tipo Primo	primo item
%Secondo 	secondo item
\begin{description}
	\item[Primo] \hfill  primo item %\hfill riempie di spazi la riga e primo item lo fa scrivere leggermente più sotto rispetto a Primo
	\item[Secondo] secondo item
\end{description}
%si possono innestare più elenchi uno dentro l'altro

